\documentclass{beamer}
%\usepackage[T1]{fontenc}
\usepackage[utf8]{inputenc}
%\usepackage{lmodern}  % Use the Latin Modern font family

\usepackage{latexsym,amsmath,xcolor,bm, amssymb, color, tikz, graphicx, amsthm, mathtools}
\usepackage{algorithm}
\usepackage{algorithmic}
\usepackage{hyperref}
\usepackage{float}     
\usepackage{CJKutf8}
\usepackage{multicol}

\DeclareMathOperator*{\argmax}{arg\,max}
\DeclareMathOperator*{\argmin}{arg\,min}
\DeclareMathOperator{\sign}{sign}
\DeclareMathOperator{\Tr}{Tr}

\makeatletter
\DeclareRobustCommand\onedot{\futurelet\@let@token\@onedot}
\def\@onedot{\ifx\@let@token.\else.\null\fi\xspace}
\def\eg{\emph{e.g}\onedot} 
\def\Eg{\emph{E.g}\onedot}
\def\ie{\emph{i.e}\onedot} 
\def\Ie{\emph{I.e}\onedot}
\def\cf{\emph{c.f}\onedot} 
\def\Cf{\emph{C.f}\onedot}
\def\etc{\emph{etc}\onedot} 
\def\vs{\emph{vs}\onedot}
\def\wrt{w.r.t\onedot} 
\def\dof{d.o.f\onedot}
\def\etal{\emph{et al}\onedot}
\makeatother


\usetheme{Madrid}
\useinnertheme{circles}


\definecolor{ColorUNR}{HTML}{0b2755} 
\usecolortheme[named=ColorUNR]{structure}
%\usecolortheme[named=ColorUNR]{exampleblock}

%\setbeamertemplate{blocks}[rounded][shadow=true]
%\setbeamercolor{block body}{fg=black,bg=white}



%------------------------------------------------------------
%This block of code defines the information to appear in the
%Title page
\title %optional
{Título de la Clase}

\subtitle{(Subtitulo de la Clase)}

%\subtitle{with applications to persuation and lie production}
% \author % (optional)
% {Author Name}

\author[Matias Ramos]{Matias Ramos}

\institute[]{Universidad Tecnológica Nacional - Facultad Regional Santa Fe}
\date[TC 2025]{Training Camp 2025}
\titlegraphic{\includegraphics[clip,height=2cm,keepaspectratio]{logos/tcarg.jpeg}}

%End of title page configuration block
%------------------------------------------------------------


%------------------------------------------------------------
%The next block of commands puts the table of contents at the 
%beginning of each section and highlights the current section:
\AtBeginSection[]
{
  \begin{frame}
    \frametitle{Outline}
    \tableofcontents[currentsection]
  \end{frame}
}
%------------------------------------------------------------


\begin{document}


%The next statement creates the title page.
\frame{\titlepage}


%------------------------------------------------------------
% Frame de Sponsors, me parece mejor ponerlo al principio
% Antes del índice/contenido

% --- Sponsors Frame 1: Organizador & Diamond Plus ---


% First sponsors frame: Organizador and Diamond Plus
\begin{frame}{Gracias Sponsors!}
    \begin{columns}[t]
        \column{0.5\textwidth}
        \centering
        Organizador\\
        \vspace{0.5cm}
        \includegraphics[width=1\textwidth,keepaspectratio]{logos/aapc.png}
        \includegraphics[width=1\textwidth,keepaspectratio]{logos/utn_santafe.png}
        \column{0.5\textwidth}
        \centering
        Diamond Plus\\
        \includegraphics[width=1\textwidth,keepaspectratio]{logos/GTSlogo.jpeg}
    \end{columns}
\end{frame}

% --- Sponsors Frame 2: Gold & Oro ---

\begin{frame}{Gracias Sponsors!}
    % Platino at the top, full width
    \centering
    Platino\\
    \includegraphics[width=0.6\textwidth,keepaspectratio]{logos/folder.png}
    
    \vfill
    
    % Gold and Oro at the bottom in two columns
    \begin{columns}[b]
        % Gold column
        \column{0.5\textwidth}
        \centering
        Gold\\
        \includegraphics[width=0.8\textwidth,keepaspectratio]{logos/neuralsoft.png}
        % Oro column
        \column{0.5\textwidth}
            \centering
        Oro\\
        \includegraphics[width=0.8\textwidth,keepaspectratio]{logos/jerarquicos.jpg}
    \end{columns}
\end{frame}

% --- Sponsors Frame 3: Aliado ---

\begin{frame}{Gracias Sponsors!}
    \centering
    Aliado\\
    \vspace{1cm}
    \includegraphics[width=0.6\textwidth,keepaspectratio]{logos/santa_fe_logo_v2.jpg}
\end{frame}


%---------------------------------------------------------
%This block of code is for the table of contents after
%the title page
\begin{frame}
\frametitle{Outline}
\tableofcontents
\end{frame}
%---------------------------------------------------------


\section{Part1}

\begin{frame}{Topic1}
Hello world! 
\begin{itemize}
    \item Introduction 
    \item Task Formulation
\end{itemize}
\end{frame}




\begin{frame}{Topic2}
\begin{itemize}
    \item $a \in $  $\mathcal{A}=$\{alice, bob, ...\}
\end{itemize}


\begin{example}[Ejemplo]
    $u_m$ can be:
    \begin{equation}
        \begin{CJK*}{UTF8}{gbsn}
        u_m = \begin{cases}
            \textrm{苹果}, & if~~a == \textrm{``apple''} \\
            \textrm{我不想回答你的问题}, & if~~a == \textrm{``refuse''} \\
            \vdots & \vdots
        \end{cases}
        \nonumber
        \end{CJK*}
    \end{equation}
\end{example}
\end{frame}

\begin{frame}{Formalization}

\begin{definition}[Def1]
A $t-1$ turn dialogue:
\begin{equation}
    H^t := \{(u_u^{1}, pg^{1}, u_m^{1}), \cdots, (u_u^{t-1}, pg^{t-1}, u_m^{t-1})\},
\end{equation}
where $pg$ is the observation .
\end{definition}

\end{frame}

\section{Part2}


\begin{frame}[fragile]{Forward Algorithm}
\begin{algorithm}[H]%[!ht]
\small
\caption{Forward algorithm.}
\label{alg:algorithm}
\begin{algorithmic}%[1]

\REQUIRE Transition model $\mathcal{T}$, value score model $\mathcal{F}$.
\STATE Compute $\bm{\alpha}^1$.
\FOR{$i = 2$ to $t-1$}
    \STATE Obtain $v_+^i$ using the model in \ref{fig:forward}.
    \IF{$i == t-1$}
        \STATE Compute $V_+(a)$ using $\bm{\alpha}^{t-1}$.
    \ENDIF
\ENDFOR
\RETURN $V_+(a)$ as a function.

\end{algorithmic}
\end{algorithm}
\end{frame}


\begin{frame}{Consultas}
Pueden consultarme durante esta semana, o me pueden enviar un mail a:
        \begin{itemize}
            \item \href{mailto:mramos@frsf.utn.edu.ar}{mramos@frsf.utn.edu.ar}
        \end{itemize}
        %\bibliography{ref}
\end{frame}


\end{document}